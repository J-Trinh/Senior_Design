Include a header paragraph specific to your product here. Customer requirements are those required features and functions specified for and by the intended audience for this product. This section establishes, clearly and concisely, the "look and feel" of the product, what each potential end-user should expect the product do and/or not do. Each requirement specified in this section is associated with a specific customer need that will be satisfied. In general Customer Requirements are the directly observable features and functions of the product that will be encountered by its users. Requirements specified in this section are created with, and must not be changed without, specific agreement of the intended customer/user/sponsor.

\subsection{Requirement Name}
\subsubsection{Description}
A detailed description of the feature/function that satisfies the requirement. For example: \textit{The box will be slate blue. This specific color is required in order to ensure that the box matches other similar boxes in the Box Systems Premium line of products. Slate blue is specified as \#007FFF, using six-digit hexadecimal color specification.} It is acceptable and advisable to include drawings/graphics in the description if it aids understanding of the requirement.
\subsubsection{Source}
The source of the requirement (e.g. customer, sponsor, specified team member (by name), federal regulation, local laws, CSE Senior Design project specifications, etc.)
\subsubsection{Constraints}
A detailed description of constraints on satisfying the requirement (e.g. one such constraint might be: \textit{The specified color must be commercially available in paint capable of adhering to the material of which the box is manufactured. (See customer requirement 3.x for production material specification.)}
\subsubsection{Standards}
A detailed description of any specific standards that apply to this requirement (e.g. \textit{NSTM standard xx.xxx.x. color specifications \cite{Rubin2012}}.)
\subsubsection{Priority}
The priority of this requirement relative to other specified requirements. Use the following priorities:
\begin{itemize}
\item Critical (must have or product is a failure)
\item High (very important to customer acceptance, desirability)
\item Moderate (should have for proper product functionality);
\item Low (nice to have, will include if time/resource permits)
\item Future (not feasible in this version of the product, but should be considered for a future release).
\end{itemize}

\subsection{Requirement Name}
Feeling
\subsubsection{Description}
The eyewear may be comfortable
\subsubsection{Source}
Dr. McMurrough
\subsubsection{Constraints}
A "comfortable" feeling is subjective.
\subsubsection{Standards}
N/A
\subsubsection{Priority}
High

\subsection{Requirement Name}
Pricing
\subsubsection{Description}
The system shall be affordable compared to competition.
\subsubsection{Source}
Dr. McMurrough
\subsubsection{Constraints}
The cumlative cost of the parts used to create the product.
\subsubsection{Standards}
N/A
\subsubsection{Priority}
High

\subsection{Requirement Name}
Portability
\subsubsection{Description}
The system shall be portable.
\subsubsection{Source}
Dr. McMurrough
\subsubsection{Constraints}
The size of the parts used to make the product.
\subsubsection{Standards}
N/A
\subsubsection{Priority}
Moderate

\subsection{Requirement Name}
Perception
\subsubsection{Description}
The system shall operate in real time.
\subsubsection{Source}
Fernando Do Nascimento
\subsubsection{Constraints}
Odroid and Cypress processing speed.
\subsubsection{Standards}
UDP/TCP/IP
\subsubsection{Priority}
High