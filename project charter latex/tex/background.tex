Students of the CSE (Computer Science Engineering) department at the University of Texas at Arlington (UTA) are required to complete a Systems Design project also called \“Senior Design\” during the final year. Dr. Christopher McMurrough was the instructor for the class. During the beginning of the semester, each student was asked to submit an introductory essay that describes their interests, strengths, weaknesses, and any project ideas if they had any. Based on our essays, Dr. McMurrough formulated our team. We had five people on our team: Fernando Do Nascimento, Joseph Trinh, Krishna Bhattarai, Zachary Allen, and James Stone
A list of project topics was suggested for the entire class. After much deliberation and thoughts our team decided to go with the Eye Tracker project.

Eye Tracking is an emerging technology that tracks the eye movement of the user. It is particularly important when it comes to tracking eye movement of people that have ALS(Amyotrophic Lateral Sclerosis) which over the time paralyses all body parts except for the eye. Eye Tracking is also used by large supermarkets to collect data of where the customers are looking at most. Based on the data they collect from the ey tracker, they rearrange items on their shelves to increase profit. Current eye tracking systems that are commercially available are very expensive. They cost as much as 20,000 or more. 

In conclusion, our goal is to create an eye tracking system, that is not only affordable but also modern looking and one that performs with higher accuracy and speed. We are looking to improve the eye-tracking technology by making the product more affordable to a wider range of users without having to sacrifice the video processing quality and precision. Another goal of our project is to create a device that is comfortable to wear, since some of the users might use this product for an extensive period of time. Finally, our main goal is to improve the quality of life of people living with disabilities such as ALS.