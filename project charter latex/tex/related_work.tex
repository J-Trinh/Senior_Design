In the past decade, eye-tracking systems have become very popular and in high demand since it has multiple applications. Some of the applications involve improving the living conditions of people living with Amyotrophic Lateral Sclerosis (ALS) or any other disease that affects the control movement of a person. Another application of the eye tracking involves marketing research for companies selling products at convenience stores or supermarkets. Today, various eye tracking systems are avaiable as open-source as well as commercial products. 

Most of the open-source eye-tracking systems available today involve using web cameras. Since, they are building a low cost system, the quality of the video being processed has a lower quality compared to the camera that we will be using with the Cypress EZ-USB CX3. Another disadvantage is that some of these projects do not use a head-mounted system, so the program is less accurate tracking the eye gaze since it is performing the algorithm from a greater distance. This means that these types of eye-tracking systems are less accurate than head-mounted eye-tracking systems and they have constraints. There are also open-source systems that use head-mounted systems. In those systems they use two types of camera modules, web cameras and a digital camera. In the web camera system, the camera is directly connected to a computer \cite{Kumar}. While in the digital camera system, a few modifications are made to the hardware but the camera is mounted in the enclosure that comes along with it \cite{Derrick}. The price of the open-source systems varies depending on the camera module to be used. 

The eye-tracking systems that are commercially available in today's market are very expensive. The most affordable system that our research showed was \$99. Other systems cost around \$22,000 \cite{Eye}. The company of the most affordable system, The Eye Tribe, provides an API for the device so the buyer may use the product for multiple purposes. This device is not a head mounted device, so it is not practical to use, since it needs to be mounted on a device or surface in order to be used. The company, EYETRACKING, which created the \$22,000 system, does not provide buyers with an API to operate the product for different uses \cite{Eyetribe}. This company provides the buyers with a software suite in order to use their product. This company provides multiple types of eye-tracking systems both head-mounted and remote. Some of the head-mounted systems seem to be very bulky and uncomfortable to wear.